\documentclass[../base_file/cs1550_notes.tex]{subfiles}

\begin{document}
\chapter{File Systems}
What is the interface to interact with the file system? Files.
\section{File System Basics}
\subsection{Files}
\textbf{Naming:}
\begin{itemize}
	\item Case Sensitive
	\item Case Insensitive
	\item Case preserving - when you reference the case you don't care
\end{itemize}
Extensions indicate structure, creation, and consumption of a file.\\\\
\textbf{Metadata}
\begin{itemize}
		\item Type
		\item Creator
		\item Structure
\end{itemize}
In desiging a file system, need to take care of data, metadata, and file
system structure.  Excetensions can be hard to change and are normally
hiddent.  Remember, a file is really just an arbitary sequence of bytes 
(most modern OS).  This was not the case in PalmOS but it was actually
records in a database.  Could also do craftier things like a tree
(Databases). As a side note, Oracle will optimize on blank hard drives.\\\\
\textbf{File Attributes:} are numerous depending on the system.\\\\
\textbf{File operations:} C file functions such as open, close, seek. 
\end{document}
