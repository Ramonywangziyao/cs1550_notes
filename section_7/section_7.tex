\documentclass[../base_file/cs1550_notes.tex]{subfiles}

\begin{document}
\chapter{Memory Management}
We can think of things we would like memory to be, but RAM is none of these.\
So why do we need to manage memory? Exclusive access so that every processes
sees a unified memory.
\textbf{Insert Picture}
\section{Fixed Partitions}
\textbf{Insert Picture}
How many partitions do we want? Also how large are the partitions going to be
so that they can hold useful programs?  The more partitions we make the more
processes can run. The goal is to maximize CPU utilization and it is unlikely
that a single process can use up all CPU time.  Very CPU bound processes may
fill RAM with only 1 or 2 partitions; however, most processes are
not this CPU bound, and therefore we can often run more processes.  The number
of processes is the \textbf{degree of multiprogramming}.\\\\
Chop up memory into enough partitions to get the desired degree of multiprogramming.
A process may be assigned a partition and be in a queue for the partition or there
could be a single global queue for all the partitions.  With \textbf{multiple queues}
you may have empty/unused partitions.  With a \textbf{singular queue} you may over
allocate and waste memory.  There may be a wait -- small in needs gets a lot of space
and runs a long time may starve processes that need a lot of memory.  Essentially,
multiple queues lead to external fragmentation whereas a single queue leads to 
internal fragmentation.
\subsection{Relocation and Protection}
Fixed size partitions are still shared between processes and there is nothing currently
to prevent a process from reading/writing into another processes memory.  When we assign
a partition, all memory should be within the bounds of the partition.  This is the
\textbf{Protection Problem}.  There is also the \textbf{Relocation Problem} of moving 
a process may change locations of instructions and the associated references.\\\\
You can't use absolute addreses in the relocation problem, but relative addresses are OK\@.
How common are absolute addresses?  Well every jump and function call is an absolute
reference and not known at compile/load time.  These problems exist due to the lack of 
exclusive access.\\\\
Let us try to fix these two problems for fixed partitions.  
\begin{itemize}
	\item Protection
		\begin{itemize}
			\item Keep track of partition bounds (min, max or min and size)
   			\item Need to check any instructions that use addresses\dots.every instruction
					does a modification (\$PC + 4) on fetch!
			\item Do checks in CPU because that is where instructions come from.  The OS
					can't do it because it does not know about addresses. Only the CPU
					is doing work (run time) and knows addresses.  There are priveleged
					registers and a need for hardware support
			\item This is only the solution for partitioned memory
		\end{itemize}
	\item Relocation Solution
		\begin{itemize}
			\item Never use absolute addressing -- which is actually a possibility
			\item We we load, rewrite addresses, but this is no fun
			\item Hardware can add base to absolute address -- the CPU should know partition
					base
		\end{itemize}
\end{itemize}
\section{Swapping Memory}
At the end of the day, fixed partitions are dumb.  Much rather have a swapping memory model.\\\\
\textbf{Insert Figure}\\\\
Imagine we have a memory scheduler, we can swap process A out to disk via an eviction.  However,
on restarting process A we want the same execution.  This leads to a \textbf{dynamic relocation
problem} because process A could return to a different base address.  Protection problem could
be solved by limit addresses.  Despite these problems, we still like swapping.  Think about
why the stack and heap grow towards each other.
\subsection{Allocation Management}
The stack is not the problem here, instead it is the heap, or the region for dynamic allocations. 
This is because there is no guarantee on the order of allocation or deallocation.  One solution 
for heap mangement is to use a \textbf{bitmap}.  However, it takes a lot of space if the chunk
size is too small and growing the chunk size leads to \textbf{internal fragmentation}.  Commonly
the chunksize is $\sim4$ Kb.  If the bitmap is sparse, because we used something other than a
unary number, compression can be an option.\\\\The other option is to use a \textbf{linked list},
but this can degenerate to be worse than a bitmap; however this rate.  The linked list can be
stored near the region where the allocation occured.\\\\
With a linked list there are different algorithms for allocation (think back to CS0449):
\begin{itemize}
	\item First Fit
	\item Quick Fit
	\item Worst Fit
	\item Best Fit -- External Fragmentation
	\item Next Fit -- Avoid rescanning
\end{itemize}
In both allocation strategies, \textbf{coalescing memory} is important in order to prevent
external fragmentation.  A bitmap handles this naturally, whereas a linked list requires some work.
Despite this overhead, a linked list is considered better than a bitmap.\\\\
\textbf{Overlays} are hand written dynamic loading of subsets of a program's code and data.  Maybe
the space given is not large enough and as a result something must be done.  \textbf{Insert Figure}\\
At any given moment we only use a part of code/data and as a result we can make partial progress. 
Consider tracing a function, we can start $g$ as soon as $f$ stops.  However, the problem is that
they might have different globals, heap, and activation record.  So we just swap in? How?  It is hard,
but does allow for rapid movement from one subset of total needs to another as subsets completes.
The subsets are overlapping and it can be a programmer's problem.  When I said it is a hard task, it 
is actually a horrible, tedious task that a computer can do for us.
We have the illusion that a process has all the space, thereby side stepping the protection problem.\\\\
In 32-bit systems, we have a total of $\sim4$ billion addresses and we now have a lot of RAM and can
therefore achieve a high degree of multiprogramming; however, no matter how much RAM we get this will
likely always be a problem.
\section{Paging}
In overlays the programmer was responsible to play variables in memory in order to keep addressing
consisten.  This is a problem in \textbf{translation}:
\begin{center}
$Virtual Address\ \longmapsto\ Physical Address$
\end{center}
\subsection{How to Map?}
One way to map is to use a \textbf{hash function}: $h(x)\ =\ x\ mod\ s$.  By the \textbf{pigeon hole
principle} we will have collisions and can have two needed resources at a single address. This is 
\textbf{prescriptive} and tells where something should be.\\\\
Alternatively, a \textbf{table} or an array can be used and is a \textbf{descriptive} solution 
so there is no need to worry about solutions.  The mapping is done by: $table[VA]\ \longmapsto\ PA$.
\subsection{Table Location}
This can't be done in the OS because  it is not running or does not have access to physical addresses.
Also the program won't do it for you.  The solution is in hardware. \textbf{insert figure}.  There are
upto three possible memory accesses per instruction:
\begin{itemize}
	\item On Fetch
	\item Looking through table
	\item Getting data from memory
\end{itemize}
\subsection{Table Size}
Physical memory may have a limit and physical addresses have a hardware dependent size.  Assume we have
a 32-bit physical address space.  There is nothing magical about the number 32. The height of our table
is therefore $2^{32}$ and the width is $2^{2}$ (a block).  This leads to 16 Gb per process\dots which is too much.
A solution is to reduce data structure size (remember the bitmap?) by reducing the number of elements.
Here the height is too big.  Divide virtual address space into chunks and move chunk sized things into RAM\@. 
This is "chunk" is what we call a \textbf{page}.\textbf{insert figure}\\\\
The page and \textbf{page frame} must be of the same size so that the mapping can be done as:
$page\_table[page\ number]\ \longmapsto\ frame\ number$.  However, did we succeed in reducing page table size ($h\times w$)
\begin{center}
	$w\ =\ 4$ (Still, we will find use for those bits)\\
	$h\ =\ number\ of\ pages\ =\ \frac{address\ space\ size}{page\ size}$
\end{center}
Using $2^{32}$ for the address space size and $2^{2}$ for the page size, the result is now 4 Mb per process.\\\\
If you have too large of a page size, the degree of multiprogramming is hurt due to less efficient subsetting.
This leads to the \textbf{cardinal sin}: Data structure size and access time are inversely proportional.  Our 
solution above suffers from this problem.\\\\
\newpage
Consider the \textbf{MMU algorithm}:\\
Input: Virtual Address, Page Table\\
Output: Physical Address\\
Steps:
\begin{enumerate}
	\item Virtual Address $\longmapsto$ Page Number
	\item Page Table[Page Number] $\longmapsto$ Frame Number
	\item Frame Number $\longmapsto$ Physical Address
\end{enumerate}
How to number pages? (0 to number of pages - 1).  Page number is $virtual\ address/page\ size$.\\
The physical address is $frame\ number\times page\ size$.\\\\
\textbf{Example:}\\\\
VA = 4000\\\\
Page Size = 4096\\\\
$\frac{4000}{4096}\ =\ 0$\\\\
Page Table[0] = 1\\\\
$1\times 4096\ =\ 4096$\\\\
\textbf{Problem:}
The virtual address 4004 also gives same mapping: $\frac{4004}{4096}\ =\ 0$.\\\\
\textbf{Solution:} Forgot $offset\ =\ virtual\ address\div page\ size$\\\\
\textbf{Insert Figure Here}\\\\
Table size is dependent on resolution of translation so we can track coarser granularity in virtual addresses by
chunking it as a page in virtual address space.  Physical addresses corresponding to a page frame and 4 Kb pages
result in a 4 Mb page table.  However, the work to  use the table may be harder now.\\\\
Looking back at the MMU algorithm, it clearly runs in $\mathcal{O}(1)$ however this is not adequate at the 
systems level.  From an achitecture point of view we have to do \textit{math}: modulo, division, multiplication.
If the page size is a power two we can use shift (compiler folks call this a strength reduction) since 4 kb = 
$2^{12}$.  Shifting right is division and shifting left is multiplication.\\\\
There is an even better way to do this by taking advantage of the hardware and doing a simple wire trick:
\textbf{Insert Figure Here}.\\\\
Although 4 Mb per process is pretty good, can we do better?  If the page table is sparse we can change the data
structures.  The question is, are all of our page table entries useful?  Also note that that the number of bits 
for frame number is less than that of the physical address. \textbf{Insert Figure Here}\\\\
If the valid bit is 1 the page is in RAM\@.  Otherwise, the valid bit is 0 and represents an invalid mapping.  The
question is are the valid bits mostly 1 or 0?  Since there is no need to map all the freespace between the heap 
and the stack the answer is \textbf{mostly 0}.  However, we have a lot more pages than page frames.
\subsection{Page Linked List}
Nodes \textbf{insert figure here}\\\\
However have to do a $\mathcal{O}(n)$ every time and every node look up is a memory access -- very expensive work.
We don't do this.
\subsection{Binary Page Tree}
Worst case is $\mathcal{O}(n)$ (degeneration to linked lists).  So we would want a self balancing tree: Red-Black
Tree, AVL tree, etc.  All you get is $depth\ =\ \mathcal{O}(log(n))$.  What can be done is to fix the tree depth
because all the numbers in a tree's runtime analysis are a balancing act.\textbf{Insert Figure Here}\\\\
If the depth is fixed, the branching will increase.  Also note that the root is only used to pick the correct leaf
node.  Say there are $r$ leaves, we have to get the same functionality as the page table.\\\\
\textbf{Some Analysis:}\\
$number\ of\ pages\ =\ r\times l$ where $l$ is the size of a leaf node.\\\\
\begin{tabular}{l*{3}{c}r}
$r$ & $l$\\
\hline\\
$1$ & $2^{20}$\\
$2^{20}$ & $1$\\
\end{tabular}\\
\textbf{Insert Figure here}\\\\
$Size\ of\ tree\ =\ Size_{root}\ +\ N\times Size_{leaf}$ where $Size_{root}\ =\ r\times 4$ and $Size_{leaf}\ =\ l\times 4$.\\\\
What makes this potentially sparse is that few leaves will exist since most page table entries (PTEs) do not have a valid
bit.  If a there is a single valid entry in a leaf, we have to have the whole leave, however.  If $r\ =\ 1$ and $l\ =\ 2^{20}$
all would have to be invalid in order to save space (N is either 0 or 1).  We don't do this.  Instead if $r\ =\ 2^{20}$ and
$l\ =\ 1$, \textbf{insert figurehere}, we just have the page table again (actually worse since there is an additional memory
access).  So let us compromise with	$r\ =\ 2^{10}$ and $l\ =\ 2^{10}$.  Root will be 4 Kb ($1024\times 2^{2}\ =\ 2^{12}$) and
leaves will also be 4 Kb.  By saving only two leaves we are smaller than the original page table. \textbf{Insert Figure Here}\\\\
As a result we need to be able to omit leaves since we have a data structure which does not require that all leaves need to
always exist.  To omit a leaf all PTEs in the leaf need to be invalid which is 4 Mb of unmapped information. Remember there
are a lot of free pages between the heap and the stack and we have the assumption of sparseness -- we are not using most
of our address space. Another issue is that the 4 Mb of invalid entries must be properly alligned to a leaf (4 Mb boundary).
Another option is to have many more levels to the tree, but this results in more unwanted memory look ups.  (\textit{Aside:
Address space layout randomization puts offsets at some page size multiple}).
\subsection{The Future is Now -- 64-bit Architecture}
\textbf{Insert Figure Here}\\
\textbf{Insert Figure Here}\\
\textbf{Insert Figure Here}\\
\begin{center}
$\sum Valid\ Pages\ \leq\ Number\ of\ Page\ Frames$
\end{center}
Sparseness exists if you look at valid pages compared to total pages -- does not if you are looking at valid pages
compared to total numbe of page frames -- this is dense.  As a result you only need to have one table across all
processes because frame numbers are shared across all processes.  This results in an \textbf{inverted page table}.
This is a move from per process to per system since address are not unique, Questions such as "Whose page 7 does it
belong to?" are not valid since every process has a page numbered 7, although different processes have different
information in their page 7.  To identify, need to add PID information.
\section{Caching}
Smallest data structure has a prohibitive cost to use (Look back at MMU algorithm). We can't have $\mathcal{O}(n)$
algorithms at the system level.  However, the trade off between size and speed is bidirectional.  As a result we can
do \textbf{caching}.  This is not \textbf{THE CACHE}.  In order to save time we can give memory to the problem whenever we
have locality.  There are two types of locality or nearness/closeness of something:
\begin{itemize}
	\item \textbf{Temporal Locality:} You will use things again, so do expensive work once and look it up again
	\item \textbf{Spatial Locality:} When you visit an element you will probably visit its neighbors
\end{itemize}
NEed to show locality in address space in VA and then show it can be exploited in MMU algorithm.  Code has spatial
temporal locality because it might exist in one page and would have the same frame number. The look up in the MMU
has temporal locality.  Can we build a cache that is fast and small? Yes--\textbf{Transition Lookaside Buffer (TLB)}.
\textbf{insert figure here}\\
\textbf{insert figure here}\\
Smallness?  Only has to be 16 or 32 entries. The mapping is done as $IPT[Frame\ Number]\ \longmapsto\ (PID,\ Page\ Number)$.
Why can the cache be so small?  To fill we would at a given instant need 16 to 32 different pages in use and this is a lot
of information.
\subsection{Software or Hardware Cache?}
A 16 entry TLB is something managable by the OS and can be changed during a context switch.  This is \textbf{Software Managed
Cache}.  A \textbf{Hardware Managed Cache} is handled by the CPU and entries are tagged by the PID\@.  Will have more than 16
entries and may be polluted with entries from different processes.
\subsection{Caching Mechanics}
When something is a in cache that we wanted, we call that a \textbf{hit}.  Otherwise, it is a \textbf{miss}.  When we have a
TLB miss have to go to IPT and search.  Misses can occur in three different cases.
\begin{enumerate}
	\item \textbf{Compulsory:} What if I have never seen a value before?  I have no choice but to find it.
	\item \textbf{Conflict:} Big caches have to do some indexing and this is likely a hash function.  What we want is gone
								due to being overwritten.  Can be eliminated in TLB due to size.  Hardware solution is to
								do parallel comparisons and MUX out the right one.  Known as \textbf{Context Addressable
								memory}.
	\item \textbf{Capacity:} Something new comes and overwrites something old -- least recently used.  We have diminishing
								returns here.  Even at inifinite size we will still have compulsory misses.  
\end{enumerate}
Yay! Our inverted page table was not a bad idea! But no one uses this! \textbf{Insert Figure Here}
What is actually done is we replace each level of the page table with a TLB step.  The TLB still makes sense, but the question
now is where do we put the TLB\@?  It can actually be fabricated directly on the processor.  This makes it worthwhile even for
a single level page table.  We are now done with how we convert virtual addresses to physical addresses. (\textit{Aside:  In the
systems world, $\mathcal{O}(1)$ is really $\mathcal{O}(c)$ for some constant $c$} and this constant can be very relevant).
\section{Paging Mechanics}
Let us take a closer look at the page table entry.  \textbf{insert figure here}.  The protection is done in 3 bits for \textbf{read},
\textbf{write}, \textbf{execute} and exist because we would like to avoid executable code on the stack (buffer overflow).  However,
there exist \textbf{just in time (JIT) compilers} that heap allocate but it is executable.  For example, /dev/null is not readable but
is executable.  The \textbf{dirty bit} makes modifications on a page, such as writes (\$sw), by being set to 1.  \textbf{Reference bit}
starts out at 1 since we want to use it when we bring it into memory.
\subsection{Page Fault}
How do we establish a mapping and bring pages into the frames?  A \textbf{page fault} is an exception raised by MMU indicated a requested
virtual address is not in memory.  When something bad happens in hardware, the hardware will raise an exception for the OS to handle.  If
there is no page in RAM, may the program did not do anything wrong, hence the page fault.  In order to handle a page fault:
\begin{enumerate}
	\item Determine the faulting virtual address
	\item If the page is invalid, grow stack/heap and segfault on error
	\item If physical memory is full, choose a frame to evict
	\item Write frame to be evicted to disk if dirty
   	\item Load requested page into now empty frame
\end{enumerate}
For $2$ the OS will get info from the executable on disk and tells the CPU to try again.  If faulting address is associated with stack
pointer or heap you have to grow things.  Though the page faulting mechanism, we figure what parts of a program you want to use, this
is known as \textbf{demand paging}.  The page faults do our subsetting (yay!).\\\\
As expected, at the beginning of a program there are going to be a lot of page faults because nothing is in RAM at the start and there
are a lot of compulsory cache misses.  There is a way to avoid the compulsory cache miss -- \textbf{prefetching} and the OS can do
something similar by \textbf{prepaging} but this is not perfect.  Not all MMU translatable addresses are valid and the OS will catch
bad addresses and send a \textbf{SIGSEGV} leading to our beloved segfault.  In C if you overun an array you have to index so badly that
you run into an invalid page.  The virtual memory subsystem will tell you made a mistake.\\\\
There are only a finite number of frames so we have two options:
\begin{enumerate}
	\item Do nothing and wait for RAM to open up for us
	\item Take a frame and give it to demanded page
\end{enumerate}
We don't want to wait and we also don't want to the memory system to make defacto scheduling decisions.  So we now have the challenge
of deciding which page to evict.  If a page is clean we don't have to make a disk write.  This occurs in two cases:
\begin{enumerate}
	\item It is a code portion that won't change
	\item We wrote it out in a prior eviction
\end{enumerate}
There are algorithms for deciding which page to evict.
\subsection{Algorithm 1: The Optimal Algorithm}
First consider the worst algorithm: evict every page that I need next would lead to a chain of page faults (the maximum page fault rate).
To get the best algorithm, just do the opposite and evict the page that I need furthest in the future from where program is currently.
This may be a page that I will never need again.  This will lead to the minimal page fault rate.  However, to implement this you need
perfect knowledge of the future (\textit{Aside: or a lot of spice}) and is therefore impossible to implement.  Any implementable algorithm
is not optimal (OPT).  All other algorithms we will consider will consider the past to predict the future.
\subsection{Algorithm 2: FIFO}
Evict page that has been in a frame (RAM) the longest.  Not a great algorithm because OS may have pages from even boot time.  Also have to
consider that some pages are \textbf{pinned} and not considered for eviction.  A lot of the pages belonging to the OS are pinned.  FIFO
also lacks a good notion of time.
\subsection{Algorithm 3: Not Recently Used (NRU)}
Evict page that is the oldest, preferring pages that are not dirty.
\begin{center}
\begin{tabular}{l*{3}{c}r}
Preference & Referenced & Dirty\\
\hline
1 & 0 & 0\\
2 & 0 & 1\\
3 & 1 & 0\\
4 & 1 & 1\\
\end{tabular}
\end{center}
\textbf{Insert Figure Here}\\\\
If looking at unreferenced pages, there might be a lot, so the algorithm narrows search space by preferring clean pages.  If a page fault
occurs early in the reset time, every page will look unreferenced.  As a result a page that is in use may get eviceted.  Also, there might
be hundreds of pages in a given category, in which case the algorithm chooses randomly.
\subsection{Algorithm 4: Second Chance}
Combine FIFO with NRU to do a little better.
\begin{center}
\begin{tabular}{l*{8}{c}r}
Ref & Ref & Ref & Not Ref & Ref & Not Ref & Ref & Ref\\
A & B & C & D & E & F & G & H\\
0 & 4 & 8 & 15 & 21 & 22 & 29 & 30\\
\end{tabular}
\end{center}
If oldest page (A) is already referenced and then selected by the algorithm, give it a second chance by resetting referenced bit and update
time.  Process A is then placed at the end of the list with R = 0 and t = 32.  Repeat until eitehr a page to evic is found (D) or you cycle
back to process A.  The problem is that there is a lot of work to be done in this algorithm, but it is good that it evicts an old unreferenced
page.  
\subsection{Algorithm 5: Clock}
\textbf{Insert Figure Here}\\
Second chance with a circular queue.  Would set A to unref and update time and then move point to B.  D would still get evicted.
\subsection{Algorithm 6: Least Recently Used}
Look to the past to predict future by keeping a time stamp.  To maintaing a queue in LRU order may require a sort/search every once or twice 
a machine instruction.  Writing a bit however is alright.  If you have time stamps a traversal is needed, but this is also ok. Getting the
time stamp is trickier to achieve.  Our working definition of time needs to have fine granularity.  Could use number of clock cycles, but a 
4 GHz processor is too fast for a 32 bit number.  Maybe a 64 bit number?  Writing this code is a problem because it grows the page table size
and is cost prohibitive.  Software implementation is not practical, so we approximate LRU by makign a compromise on bit size.
\subsection{Algorithm 7: Aging Scheme}
Use 8 bits and use R bit.  A refresh does not throw away information.  What is done is to take current R bit and shift logical right them
into counter and reset R bits.  Our counters keep track of the history.  Eviction is done by removing the page with the smallest counter value.
The high order bits of a small number are zero.  We use numeric value instead of leading zeros in order to break ties.
This is a decent approximation to LRU\@.  The 
main difference is that LRU can look back in time infinitely whereas Aging is limtied to the counter bit width.  Note: there are always n+1 bits
in the width because the R bit is really the MSB\@.  Won't evict a page if it has just been referenced.
\subsection{Algorithm 8: Working Set}
\textbf{Insert Figure Here}\\
The working set is the set of pages we are currently using.  $W(k,t)$ is the size of the working set at time time $t$ this is a union of pages
used.  An overlay is a manual redefining of the working set over time.  Demand paging discovers the working set whereas the prepaging tries to
guess the working set.\\\\
The working set changes over time, and given the working set a good algorithm would be able to evict pages outside the working set.  We have
been trying to do this with our other algorithms.  Since the other algorithms do not have an exact idea what you evict they are very conservative.
If you know what is the working set you can evict the complement of the working set -- everything not in the working set.  On next page fault
we will have free frames and have no reason to run the eviction algorithm.  However, there is a drawback to kicking out the complement of the
working set since you loose the ability to get lucky.\\\\
The working set is dynamic and it not often known apriori.  As a result we have to define a notion of time known as the \textbf{Current Virtual
Time}.  The length of time we will consider for a reference to be in the working set will be defined as $\tau$.  There is a good figure in the
Tennenbaum book.
The notion of evicting a dirty page is not considered since disk writes are very slow.  We can do the writes and searches in parallel.\\\\
There are several ways we can improve upon the working set algorithm:
\begin{itemize}
	\item \textbf{Early Exit} -- search until a page that is clean and older than tau.  Drity pages can be scheduled for write outs as they are
		discovered.
	\item If not found, a disk write will eventually complete so we can evict a written out dirty page not in the WS.
	\item Ideally after writing out we can just evict.  This is more aggressive but reasonable.  Alternatively we can just makre it clean.
	\item Just pickup where the algorithm left off
\end{itemize}
\subsection{Algorithm 9: Working Set Clock}
Implementing the above changes to the working set algorithm.  Really a hybrid of working set and some other ideas.
\subsection{Related Stuff}
\textbf{Belady's Anomaly:}  In a FIFO system an increase to ram may increase page faults.  This occurs if it is simulatable with a stack.\\\\
The unreasonable effectiveness of cache terminology in memory is because RAM is really a cache.
\section{Local and Global Allocation}
In demand paging if we load a new process and see that RAM is full we don't get anything and our page table is empty.  We can scan every
processes page table or we could use an inverted page table.  How to avoid this?  Allocate a minimum number of frames on the start of a 
process.  Page replacement will get one of these minimum.  Can't increase this number, a lot of page faults if working set is greater than
number of frames.  This leads to lots of disk writes and slow down.  If a process does not fit we \textbf{thrash}.  In \textbf{Global
Allocation} we can put our pages in a frame stolen from another process.  In \textbf{Local Allocation} we replace one of our frames.  The
main difference is that global allocation allows our working set to grow.  Growing without bound and releasing could be a problem, but the
working set can also shrink.  Global allocation makes this easy  because another process will just take the frame from you).  If you don't
want to search or have an IPT, local allocation is done.\\\\
Creating a process requires frames.  The OS could keep a reserve of empty frames which it collects from exited processes.  OS could observe
page fault frequency to determine when to grow the working set of a process. \textbf{Insert Figure Here}.\\\\
If page fault rate is too low can take away frames from a process to get it closer to compulsory page fault frequency.
\section{Page Sharing}
The mystery of static/dynamic linking and the mystery of MMAP and the mystery of fork().  How do we clone so fast? (Please don't time fork()).
Map multiple pages to a single frame, MMAP took a MAP\_SHARED argument.  Portion of parent/child were backed by the same frame in RAM.
When  you want libc you don't reference libc because someone else did that.  You reference that page with a link loader -- Dynamic Linking.
\subsection{Fork}
Fork allows you to make a new process table entry with a new page table that is a duplicate of the parent.  At some point parent/child will
write and should have exclusive address spaces.  When cloned, mark all pages as READ\_ONLY.  When either the parent or child write to a page
the OS will realize this and duplicate necessary modified page.  This is known as \textbf{Copy On Write (COW)}.  You might have noticed this
when working with QEMU and the cow2 file.\\\\
When a disk write occurs depends upon the algorithm.  Windows keeps a swap file where as unix/linux has a swap partition.  Swap area ca be
on disk because a lot of the address space is not used.  If we only write out pages that were once in RAM, less disk is requried.  However
another table needs to be built.
\section{Other Memory Details}
\subsection{ID Spaces}
Lack of address space can lead to a shortage of virtual address.  What is done in this case is to split code and data address space.  Can
crate two separate address spaces.  Use same address sapce for either code or Data but need to recognize the instruction type.  There is a 
loss of the ability to create self modifying code.
\subsection{Segmenation}
In a 16 bit system can only have 64 Kb. As a result make further devisions beyond ID.  Given an address, contextualize it into a segment.
Processor allow for segment picking and instructions were relative to a segment (ideas of near and far).  The flat memory model has a single
segment and is easier to code in.  In 64 bit systems segmentation is essentially gone.  This is a way of presenting a programmable interface
to a physically realizable memory.
\end{document}
