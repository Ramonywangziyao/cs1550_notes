\documentclass[../base_file/cs1550_notes.tex]{subfiles}

\begin{document}
\chapter{Interprocess Communication}
\section{Preliminaries}
On a single processor system we can run at least two processes.  When we
have parallelism, it creates its own problems (ie. The Race Condition).\\

\textbf{Note:} \textit{mmap} can be used to share an adress space between
parent and child processes (See Project 2).\\

Preemption is trigged by a hardware interrupt -- at any machine instruction
boundary an interrupt can go off (For the more curious look up
\textbf{precise} and \textbf{inprecise} interrupts.\\

Our job is to find regions of code where race conditions can exist --
\textbf{critical regions}.\\

\textbf{Example:}\\
\begin{lstlisting}
enter_critical_region(); // Mutex or Semaphore?
	A[tail++] = 20;			
leave_critical_region(); // Mutex or Semaphore? 

enter_critical_region() {
	disable_interrupts(); // This is wrong because it can be abused
						  // It is too coarse..preemption is not the
						  // real reason for race conditions
}

leave_critical_region() {
	enable_interrupts();
}
\end{lstlisting}

Imagine good pseudo-parallelism: $A\ \rightarrow\ C\ \rightarrow\ A$.
C has nothing to do with A's shared data.\\

What we want:\\
\textbf{Insert Figure Here}\\
\textbf{Goals:}
\begin{itemize}
	\item No two processes should be in CR at the same time
	\item No assumptions about cpu speed or number of CPUs
	\item No process outside its critical region may block another
			process (subtle point)
	\item No process should have to wait for ever to enter its
			critical region
\end{itemize}
These are the goals of \textit{enter\_critical\_region()} and \textit{leave\_critical\_
region()}
\section{Implementation 1: Strict Alternation}
\textbf{Example:}\\
\lstset{language=C}
\begin{lstlisting}
/* This is from Process A Perspective */
int turn;

enter_CR() {
	while(turn != 0); /* Or 1 for Process B */
}

leave_CR() {
	turn = 1; /* Or 0 for Process B */
}
\end{lstlisting}

This does give us mutual exclusion and is known as \textbf{busy waiting}.  Condition does
not change unless process leaves CR.\\

\textbf{INSERT FIGURE}\\

\textit{Aside: See priority inversion problem.}\\

Enforced alternation is a problem, so lets modify the above code a bit!\\

\begin{lstlisting}
locked = 0;
enter_critical_region() {
	while(locked);
	/* There is no notion of possesion and no forced turn 
	 But this does not work! Since we can be pre-empted 
	 here. Can salvage the problem via hardware!  	
	*/	
	 locked = 1;
}

leave_critical_region() {
	locked = 0;
}
\end{lstlisting}

\textbf{Hardware Support} can come from a single machine instruction (TSL, XCHG). For
more information see Tannembaum.  The problem win our previous lock system is preemption
between test and set.  Can there be a software solution to mutual excluson - Yes...
\textbf{Peterson's Solution}.
\newpage
\section{Peterson's Solution}
Share a global state with an array of size N for N processes. We will do 2. Consider
if a process is interested and who requested last.

\begin{lstlisting}
enter_critical_region(int process) {
	other = 1 - process;
	interested[process] = true; /* you are interested */
	last_request = process; /* without this you have a race condition (live lock) */
	/* a live lock is when two processes just pass possesion neither doing any work 
	 the last_request line is needed to break the tie */
	while (interested[other] == true && last_request == process);
	/* Conditional asks: Did i make the last request? */
}

leave_critical_region(int process) {
	interested[process]= false; /* no longer interested */
}
\end{lstlisting}
Peterson's solution works by exploiting the race condition.  If the last request was
not me, there was a preemption.  I won if the other process overwrote my variable. I
made first request if other made last. The key is that enter/leave cr functions are
user space implementable mutual exclusion [Solves the same problem].

\section{Producer and Consumer Problem}
Shared buffered of fixed size.  A potential race conditions exists in changing the 
counter (c++ and c--).
\lstset{language=[mips]Assembler}
\begin{lstlisting}
lw			$t0, 0(Address of Counter)
addi		$t0, $t0, 1
#PREMPTION
sw			$t1, 0(Address of Counter)
\end{lstlisting}	
Counter can go from 3 to 4 to 4 as a result.  In x86 one can do INC[counter]. More
interestingly is the issue of a lost signal.  Consumer runs first -- empty buffer
and should sleep().  Can be preempted right before we sleep.  Producer runs, makes
1 , says wake up consumer...but signal is lost since the consumer is awake.  Producer
fills  buffer, sleeps.  Consumer is scheduled, sleeps.  Both are a sleep and we have
\textbf{Dead Lock}.\\

Race conditions occur when there is a multiple step update of shared global state.\\

\textbf{How to (maybe) fix this:}\\
\lstset{language=C}
\begin{lstlisting}
/* Consumer */
enter_critical_region();
if (counter == 0)
	sleep();
leave_critical_region();

/* Producer */
enter_critical_region();
if (counter == n)
	sleep();
leave_critical_region();

enter_critical_region();
	counter++;
leave_critical-region();
\end{lstlisting}
\textbf{This is even worse:} guaranteed deadlock (sleep while holding lock) -- condition
variable. pthread\_cord\_signal will also reacquire mutex.  Instead of blaming sleep() you
can blame wake().  Keep a memory of wake ups and it becomes a problem in accounting. An
int can keep track of this.  This gives a \textbf{semaphore}.
\section{Semaphores}
Returning to the producer/consumer problem if you are already awake and get a weake up,
maybe we should keep track of weake ups and sleeps.\\
If value is positive then wake ups
without being asleep.  If subtraction yields zero or negative, no one was trying to
wake up so I should sleep (\textbf{down}).  The negative cardinality of value is
the number of sleepers.\\
If value on add is 0 or less, someone should wake up.  This can
be an arbitrary decision.  The length of queue is the number of negatives. This can be
thought of as the following conditional:
\begin{lstlisting}
if (value <= 0)
	process = dequeue(process_array);
wake(process);
\end{lstlisting}
Put all of this into an object called a semaphore.  They were invented by Dijkstra.\\
We can use 3 semaphores to solve producer/consumer problem.
\begin{itemize}
	\item empty(n) -- initialized by consumer
	\item full(0)
	\item mutex(1)
\end{itemize}

The mutex is used for mutual exclusion.  Mutex.down and mutex.up surround our CR.
Initially mutex has 1 -- one missed wakeup.  Start with down -- we start our mutex at 1
which makes that which runs first enter first.  Mutex.down is a subtraction and a potential
sleep().  Lets say we get preempted.  Consumer runs -- downs mutex to -1 and sleeps. Producer
runs again finishes CR ups mutex to 0 and therefore someone wakes up -- the consumer.  Consumer
can now run.\\

Empty(n) and Full(0) -- the value of a semaphore can be though of as the number of compies
of a resource we can access simultaneously.  Example lab with 5 printers, first 5 print jobs
go and the 6th job has to wait.  Empty(n) means I have n empty spaces [empty space is a resource
consumed by something made].  Producer makes full spaces and consumes empty spaces.  Consumer
makes empty spaces and consumes full spaces. The interpretation between mutex and our empty
and full are the same.  Our 1 in mutex is number of access to critical region -- only 1 process
can access our critical region.\\

Semaphores are tricky -- if you swap mutex lines you get a deadlock:
\begin{center}
\begin{multicols}{2}
mutex.down\\
empty.down\\
producer\\
\columnbreak
mutex.down\\
full.down\\
consumer (start here)\\
\end{multicols}
\end{center}
The result is that we can do mutual exclusion/synchronization without busy waiting.

\subsection{Other synchronization primitives}
\textbf{Counting semaphore:} A simplified version of a semaphore that can only be locked or unlocked.\\
\textbf{Binary Semaphore:} Semaphore that only takes the values 0 or 1 (a single bit).  This is not a
mutex.  The first process to up will wake everyone else up. Also known as a \textbf{barrier}.\\
\textbf{Monitors:} Automate the process of acquisation of locks and bundle everything into one object.
Original way to get synchronization in Java (eww) is to make a synchronized method.  A method in Java
being synchronized has to check lock on object.  For producer/consumer while(true) needs to be outside.\\

There is a unifying bassis for all of our synchronization primitives -- some amount of shared memory and
uniform access.\\

Not all computers have this (distributed systems).  Own OS, scheduler, etc.  Can't depend on synchronization
for reading/writing.  Need to tell some how about CR status.  Networks are unreliable.\\

Some solutions to this problem are:
\begin{itemize}
	\item Elect a computer the leader.  Shared state is moved from local to shared.
	\item Have nodes vote if neighbors in critical or if they entered.  Ask only about $\sqrt{n}$
\end{itemize}





\section{Why the fuss?}
Why are we talking about this in operating systems?  The semaphore can have a race condition
with itself.  You can't solve this with one huge hardware instruction (CISC instruction). Can't
wrap this in petersons -- you sleep with lock.  Condition vars can be used, but it is the same
problem.  The shape of the problem is the same. You can, however, disable interrupts.\\

Move up() and down() into the operating system so it is the OS that is putting processes to sleep.
In linux you will spinlock (not disable interrupts).  OS code will/should run quickly so spin locking
is not a problem.  The spin lock is also around a very small amount of code -- a best practice.

\end{document}
