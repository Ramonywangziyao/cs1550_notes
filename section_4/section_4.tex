\documentclass[../base_file/cs1550_notes.tex]{subfiles}

\begin{document}
\chapter{Scheduling}
\section{How to pick?}
Scheduling is the process of choosing which of the \textbf{READY}
processes/threads get to run next.\\

We can use the CPU intensely or be more I/O bound as a process:
	\begin{itemize}
	\item A CPU intensive process is \textbf{cpu bound}
	\item An I/O bound process will do a burst of CPU work and then
			I/O
	\end{itemize}
Note: Both types of processes can have the same \textbf{wall clock
run time}\\

If two process are CPU bound back to back, it will take $2t$ where
$t$ is 1 process time\@.  It is hard/impractical to inerleave two
CPU bound processes because switching processes is not free\@. If
you dont switch properly you loose illusion of independent program.\\

A cpu bound program wil llikely get pre-empted often.
This leads to loosing a processors time due to the switch.
Running two CPU intesive programs at the "same" time may not be better
than sequentially running.
As mentioned above, it is not easy to schedule a bunch of CPU bound
processes together.\\

In a premptive system, a process can get preempted and end up exiting
(unlikely) or being blocked (I/O bound block a lot).\\

If the system is careful with the preemption timer, an I/O bound process
may never see preemption -- it will automatically block because it
has to do I/O.\\

Can overlap CPU bursts and I/O blocks of two processes.\\
It only makes sense if I/O bound processes are much greater in number than
CPU bound processes.
\textbf{Reasonable: interactive programs tend to be I/O bound.  Amhdahl's Law}.\\

Over time CPUs increase exponentially in speed (See Moore's law..although the 
quantum limit is rapidly being reached).
On the other hand, I/O speeds grow linearly.
As a result, a CPU bound program will in "time" become an I/O bound process
because we gain the ability to do a lot of computation faster than we can
read from a disk.\\
\pagebreak
\subsection{When to schedule?}
	\begin{multicols}{2}	
	\begin{itemize}
	\item Software
		\begin{itemize}
		\item Process Creation
		\item Process Exit
		\item Blocked
		\end{itemize}
	\columnbreak
	\item Hardware Interrupts
		\begin{itemize}
		\item I/O interrupt
		\item clock interupt
		\end{itemize}
	\end{itemize}
	\end{multicols}

\subsection{Where to schedule?}
We assume a \textbf{Von Neumann Architecture} where a process can be prevented
to run by denying it RAM\@.
Once a process is in RAM, CPU scheduler picks a process.\\

\subsection{Classes of scheduling}
\textbf{Admission Schedule} selects jobs to go into RAM\@.
In an interactive system, this might be the user.\\

If admission scheduler did not do a great job, there is also a \textbf{Memory 
Scheduler} which can kick out a job from RAM\@.
The problem is that a process may have made progress before the memory scheudler
gets to it.
The process has state that has to be saved.  Therefore the memory scheduler
will do a "temporary" eviction and hold the state of the process on disk.
When system is better, the memory scheduler can bring process and its
state back into RAM.\\

\textbf{CPU scheduling} -- Batch scheduling can be used for non-interactive
systems for jobs that can run overnight.
(I like to think of scheduling jobs running on a cluster - Cyrus).
This type of system can be preemptive.\\

\section{Scheduling Algorithms}
\subsection{Metrics}
\textbf{Throughput} -- $Number\ of\ Jobs / Unit\ Time$ or in Frequency (Hz)\\
\textbf{Turnaround Time} -- Time from job submission to job completion.
This is \textbf{not} execution time!
The execution time is the time a process has \textbf{available} to run.\\
\textbf{Average Turn Around Time} -- Average of all turn around times for a 
set of jobs.\\
\textbf{Fairness} -- comparable processes get comparable service\@.
Of course, comparable is not really defined\@.
You can be egalitarian and say all processes are created equal\dots but are they?\\
\textbf{Big-O} -- Asymptotic worst case run time\@.
The key is that it is \textbf{Asymptotic}.\\
\textbf{Implementation difficulty} -- we like easy things even if they are only
close to perfect\@.
Easy is also relative and hard to define.\\

\subsection{Algorithm 1: First Come, First Serve}
\begin{center}
\begin{tabular}{l*{4}{c}r}
Queue		& 4 & 3 & 6 & 3 & Runtime\\
			& A & B & C & D & Process\\
\end{tabular}
\end{center}
After FCFS (first come, first serve):\\
\begin{center}
\begin{tabular}{l*{4}{c}r}
Queue		& 4 & 3 & 6 & 3 & Runtime\\
			& A & B & C & D & Process\\
\end{tabular}
\end{center}
Such change, much wow!
Also this algorithm is $\mathcal{O}(1)$ -- so yay!\\
Throughput: $4\ jobs/16\ time = 1/4$\\
Average turn around: $40/4 = 10$\\
\begin{center}
\begin{tabular}{l*{4}{c}r}
Process		& Completion	& Arrival	& $\bigtriangleup t$\\
\hline
A			& 4				& 0			& 4\\
B			& 7				& 0			& 7\\
A			& 13			& 0			& 13\\
D			& 16 			& 0			& 16\\
\hline
Sum			&			    &	 		& 40\\
\end{tabular}
\end{center}
No matter what is the schedule, the through put is the same.
We don't consider reality and delays.
\end{document}
