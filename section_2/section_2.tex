\documentclass[../base_file/cs1550_notes.tex]{subfiles}

\begin{document}
\chapter{Begin Multiprogramming}
Or juggling processes
\section{How to support multiprogramming}
\textbf{Can I divide RAM?}
	\begin{itemize}
	\item A process could access an address that is not in its RAM block
	\item Protect a block and check?
	\end{itemize}
Where is the best place to do checks?  \textbf{Hardware can help perform OS tasks}
Do I pack tightly in memory?  Many Processes but there are issues with \textbf{dynamic allocation}

\section{A Process's Address Space}
\textbf{INSERT FIGURE OF ADDRESS SPACE}\\
All address in range are your process's. None belong to another process.\\
A large region to support dynamic allocation.\\
Exclusive Access \textbf(This is not real)
	\begin{itemize}
	\item Does not make sense for same reason as dividing RAM -- we don't have enough resources
   	\item \textbf{The lie of virtual memory}
	\end{itemize}	
The OS is a layer between user and resources.  Insert CS0449 Diagram.  We make system calls from userspace
to kernel space.  All requests for a resources must go through the OS -- you should \textbf{not} be able
to side step OS.
	\begin{itemize}
	\item Hardware provides us with a Partioned Instruction Set (Some for Userspace, Some for Kernel Space)
	\item Some instructions are safe (add int)
	\item Some instructions are priveleged (kernel mode) -- Idea of a \textbf{mode bit}
	\end{itemize}
If you try a priveleged instruction in user mode, an \textbf{exception} is raised and OS sends a
\textbf{signal} and terminates the process.\\

Syscalls are mechanistically different than function calls -- mode changes.  Can't jump and link to our
different address space.\\

A syscall is an interrupt.  In x86 you put a trap in eax.  These are \textbf{software interrupts}. Later
we will talk about \textbf{hardware interrupts}.\\

\textbf{Interrupt Vector} -- Indexted by ints.  When interrupt occurs, goes to correct index.  Gets
address (to syscall handler?)\\

\textbf{Syscall table} -- grab new address for code of the syscall\\

OS only needs priveleged mode to change machine state.  Not everything in OS is priveleged.\\

The OS is not the same as other processe.  It does not compete for CPU time.  It does not schedule itself.
It is basically \textbf{pure overhead}.\\

The OS does not need to exist, but we are afraid a process may misbehave.  As a result, the OS exists out
of practical necessity.  We don't really want this, but we have code (OS) and it needs resources. However,
the OS only runs when it needs to: Reacting to events.  This takes time.\\

Think back to CS0447 and assembly.  On a function call, everything had to be returned to the original
state.  A clean up needed to be done.  Similarly, the OS needs to make room for its code.
The \textbf{caller context} state will have to be saved and restarted.  \textbf{A context switch}.\\

A syscall does not save context.  \textbf{The OS does it before the syscall}.  Context will be saved it
to RAM and put at the top of the caller's stack.
	\begin{itemize}
	\item Safe
		\begin{enumerate}
		\item Code was interrupted (caller) -- can't execute until OS returns
		\item RAM is a shared resource as a whole -- address space abstraction, pieces of memory are
			  mine
		\end{enumerate}
	\end{itemize}
We believe that a single context switch is \textbf{optimized}  (from a hardware/software end).  Only
way to go faster is to have fewer context switches. If we have two solutions and one uses \textbf{fewer}
context switches, we will say it goes \textbf{faster}.\\

\textbf{Resources to protect and share:}
	\begin{itemize}
	\item CPU Time -- Preemption
	\item Memory -- Virtual Memory
	\item I/O -- Spooling
	\item Security -- \textit{'Tis black magic\ldots' (Cyrus)}
	\end{itemize}
\textbf{Memory trade off -- cost, speed, capacity}\\

There are also hardware interrupts.  However the actual action is that of a software interrupt.  Think
about a \textbf{bus signal}.  It has some basic steps that allow the OS to react.

\section{OS Design Schemes}
There are two big types of OS Designs: \textbf{Monolithic} and \textbf{Microkernel}

\subsection{Monolotic OS}
INSERT FIGURE HERE\\
Think about the OS as another application which controls everything.  \textbf{Monolithic Design} is how
we normally write an application.  In this design, the OS is priveleged.  Consider scheduling: need a 
data structure, scheduling algorithm.  This is a lot of code.  In a monolithic design all of this can
be done \textbf{without privelege}!  Priveleged instructions came when you make context switch, set up
memory space, etc. This low level state is is not doable by unpriveleged instructions.\\

\textbf{In a Monolithic OS:}
	\begin{itemize}
	\item Code to maintain scheduler code
	\item Priveleged code to do context work
	\end{itemize}
All of this bundled together in a monolithic OS.

\subsection{Microkernel OS}
INSERT FIGURE HERE\\
In comparison, a microkernel OS strives to pare down OS size by extracting unpriveleged code into separate
processes.  \textbf{Servers} communicate with the microkernel to ge right answer.  The microkernel then
goes back and acts on it.

\subsection{Which is better?}
\textbf{It depends}.
	\begin{itemize}	
	\item Context Switches: The microkernel makes more context switches.  The monolithic kernel only needs
			to make 2.
	\item Code surface: The microkernel has less code -- smaller attack surface.  Also less code is easier
		    to validate.  Therefore system wide effects are less likely.
	\item Crashing: When a crash occurs the OS runs last.  In a microkernel if a user server crashes, just
			need to pick another server.
	\item Speed: \textbf{A monolithic kernel is FAST}	
	\end{itemize}
Linux is a monolthic kernel.  Windows (NT Line) is a sort of microkernel/monolthic hybrid.


\end{document}
